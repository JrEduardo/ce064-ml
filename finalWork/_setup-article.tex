\usepackage{lipsum}

%% ======================================================================
%% Pacotes usuais

\usepackage[brazil]{babel}
\usepackage[T1]{fontenc}
\usepackage[utf8]{inputenc}
\usepackage{fourier}
\usepackage{multicol}
\usepackage{tikz}
\usepackage{mathtools} %% Funcionalidades (como \dcases)
\usepackage{dsfont}    %% Para \mathds{1} Indicadora
\usepackage{setspace}
\usepackage{tocloft}
\usepackage[bottom]{footmisc}
\usepackage{tabularx}
\usepackage{threeparttable}
\usepackage{multirow}
\usepackage{float}

\setlength{\cftbeforesecskip}{6pt}

\DeclareMathOperator{\Ell}{\mathcal{L}}

%% ======================================================================
%% Cores para links
\definecolor{url}{HTML}{000080}
\definecolor{run}{HTML}{4A0082}
\hypersetup{colorlinks, allcolors=., urlcolor=url, runcolor=run}

%% =======================================================================
%% Define negrito para os ambientes monoespaçados
\definecolor{shadecolor}{RGB}{230,230,230}

\usepackage{etoolbox}
\makeatletter
\preto{\@verbatim}{\topsep=1pt \partopsep=1pt }
\makeatother

\let\oldv\verbatim
\let\oldendv\endverbatim

\def\verbatim{\par\setbox0\vbox\bgroup\oldv}
\def\endverbatim{\oldendv\egroup \hspace*{-1ex}
    \fboxsep2pt \noindent\colorbox[gray]{0.8}{\usebox0}\par}


%% =======================================================================
%% Cabeçalho das páginas (era pra diferenciar entre páginas pares e
%% ímpares, porém não está.)
\usepackage{fancyhdr}
\pagestyle{fancy}
\renewcommand{\sectionmark}[1]{\markright{#1}}
\fancyhead[RO]{Introdução ao Machine Learning}
\fancyhead[LO]{\rightmark }
\fancyhead[LE]{Introdução ao Machine Learning}
\fancyhead[RE]{\rightmark }

%% =======================================================================
%% Define o título na página inicial
\newcommand{\horrule}[1]{\rule{\linewidth}{#1}}     % Horizontal rule
\title{
  % \vspace{-1in}
  \usefont{OT1}{bch}{b}{n}
  \normalfont \normalsize
  \textsc{CE064 - Introdução ao Machine Learning} \\ [20pt]
  \horrule{0.5pt} \\[0.2cm]
  \huge Estratégias para Classificação Binária \\
            Um estudo de caso com classificação de e-mails
  \horrule{2pt} \\[0.5cm]
}